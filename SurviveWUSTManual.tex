\documentclass{beamer}	% Compile at least twice!
%\setbeamertemplate{navigation symbols}{}
\usetheme{Warsaw}
%\useinnertheme{rectangles}
%\useoutertheme{infolines}
\useoutertheme[title,section,subsection=true]{smoothbars}
 
% -------------------
% Packages
% -------------------
\usepackage{
	amsmath,			% Math Environments
	amssymb,			% Extended Symbols
	enumerate,		    % Enumerate Environments
	graphicx,			% Include Images
	lastpage,			% Reference Lastpage
	multicol,			% Use Multi-columns
	multirow,			% Use Multi-rows
	pifont,			    % For Checkmarks
	stmaryrd			% For brackets
}
\usepackage[UTF8]{ctex}


% -------------------
% Colors
% -------------------
\definecolor{UniOrange}{RGB}{212,69,0}
\definecolor{UniGray}{RGB}{62,61,60}
%\definecolor{UniRed}{HTML}{B31B1B}
%\definecolor{UniGray}{HTML}{222222}
\setbeamercolor{title}{fg=UniGray}
\setbeamercolor{frametitle}{fg=UniOrange}
\setbeamercolor{structure}{fg=UniOrange}
\setbeamercolor{section in head/foot}{bg=UniGray}
\setbeamercolor{author in head/foot}{bg=UniGray}
\setbeamercolor{date in head/foot}{fg=UniGray}
\setbeamercolor{structure}{fg=UniOrange}
\setbeamercolor{local structure}{fg=black}
\beamersetuncovermixins{\opaqueness<1>{0}}{\opaqueness<2->{15}}


% -------------------
% Fonts & Layout
% -------------------
\useinnertheme{default}
\usefonttheme{serif}
\usepackage{palatino}
\setbeamerfont{title like}{shape=\scshape}
\setbeamerfont{frametitle}{shape=\scshape}
\setbeamertemplate{itemize items}[circle]
%\setbeamertemplate{enumerate items}[default]


% -------------------
% Commands
% -------------------

% Special Characters
\newcommand{\N}{\mathbb{N}}
\newcommand{\Z}{\mathbb{Z}}
\newcommand{\Q}{\mathbb{Q}}
\newcommand{\R}{\mathbb{R}}
\newcommand{\C}{\mathbb{C}}

% Math Operators
\DeclareMathOperator{\im}{im}
\DeclareMathOperator{\Span}{span}

% Special Commands
\newcommand{\pf}{\noindent\emph{ref. }}
\newcommand{\ds}{\displaystyle}
\newcommand{\defeq}{\stackrel{\text{def}}{=}}
\newcommand{\ov}[1]{\overline{#1}}
\newcommand{\ma}[1]{\stackrel{#1}{\longrightarrow}}
\newcommand{\twomatrix}[4]{\begin{pmatrix} #1 & #2 \\ #3 & #4 \end{pmatrix}}


% -------------------
% Tikz & PGF
% -------------------
\usepackage{tikz}
\usepackage{tikz-cd}
\usetikzlibrary{
	calc,
	decorations.pathmorphing,
	matrix,arrows,
	positioning,
	shapes.geometric
}
\usepackage{pgfplots}
\pgfplotsset{compat=newest}


% -------------------
% Theorem Environments
% -------------------
\theoremstyle{plain}
\newtheorem{thm}{Theorem}[section]
\newtheorem{prop}{Proposition}[section]
\newtheorem{lem}{Lemma}[section]
\newtheorem{cor}{Corollary}[section]
\theoremstyle{definition}
\newtheorem{ex}{贡献者}[section]
\newtheorem{nex}{Non-Example}[section]
\newtheorem{dfn}{小节}[section]
\theoremstyle{remark}
\newtheorem{rem}{Remark}[section] 
\numberwithin{equation}{section}


% -------------------
% Title Page
% -------------------
\title{\textcolor{white}{武汉科技大学生存手册 Preview1.0}}
\subtitle{\textcolor{white}{Survive WUST Manual}}  
\author{Project group}
\date{August, 2024} 


% -------------------
% Content
% -------------------

\begin{document}


% Title Page
\begin{frame}
\titlepage
\end{frame}



% 序言
\section{序言}



% 序
\subsection{序}
\begin{frame}

	\begin{dfn}[§ 1.1]
		序
	\end{dfn}
	
\end{frame}



% Definitions & Examples
\begin{frame}
	\begin{LARGE}
		\textbf{Most men lead lives of quiet desperation. }
	\end{LARGE}
	
	\vspace{1cm}

	对于中国学生而言,理想的激情最热烈的时候,往往是在高中阶段——每一个人都会有自己的一个目标,每个人都会在某个深夜和他或她畅聊自己的人生理想。那时,我们觉得,一切尽在掌控之中,觉得未来充满期待。
	\vspace{0.5cm}

	谈论理想的人很多,但实现理想的人很少。到了大学,很多人都逐渐褪去了青春的色彩,放下了对生活的期待,生活在一种“平静的绝望”中。
	\vspace{0.5cm}
\end{frame}

\begin{frame}
	\textit{大四的穿着学士服到处乱拍,大三的纹丝不动地在自习室里考研,不考研的在睡觉,大二的在汗流浃背地复习期末考试和四六级,大一的追着老师屁股后面要老师划重点。}
	\vspace{0.5cm}
	
	\textit{晚上一大四的摔着酒瓶怀念青春,大三的看着理想在彷徨,大二的通宵打LOL和Apex,大一的一人一个手机在聊QQ。}
	\vspace{0.5cm}
	
	\textit{这就是六月的大学。}
	\vspace{0.5cm}
	
	\textit{还有高三的那群傻孩子赌咒发誓地往这里冲,梦想着来大学创造辉煌。可当离开大学的时候多少人都堕落得连自己都不认识了,多少人后悔光阴似箭,多少人被伤得身心俱疲只能借酒消愁。}
	\vspace{0.5cm}
	
	\textit{—— 《这就是大学》}
\end{frame}

\begin{frame}
	\begin{LARGE}
		\textbf{从一开始就应该改变}
	\end{LARGE}
	\vspace{1cm}

	《上海交通大学生存手册》中的这两句话振聋发聩——
	\vspace{0.5cm}

	\begin{enumerate}
		\item “各位同学们,在本书的开始,我不得不遗憾地告诉大家一个消息。国内绝大部分大学的本科教学,不是濒临崩溃,而是早已崩溃。”
		\item “我看到无数充满求知欲、激情、与年轻梦想的同学们,将要把自己的四年青春,充满希望与信任地交给大学来塑造。这使我心中非常不安。”
	\end{enumerate}
	
\end{frame}

\begin{frame}
	为什么写下这本手册呢?
	\vspace{0.5cm}
	
	原因之一就是在阅读SurviveSJTUManual时受到了不小的震撼。在后来的学习生活中也经历了一些事情让我对这些深有感触。写下这本手册,并不是要教大家如何改变,而是希望给大家多一点改变的勇气。
	\vspace{1cm}

	\textit{“我步入丛林,因为我希望生活得有意义,我希望活得深刻,并汲取生命中所有的精华。然后从中学习,以免让我在生命终结时,却发现自己从来没有活过。”}
	\vspace{0.5cm}
	\textit{——梭罗《瓦尔登湖》}
	\qed
\end{frame}


\subsection{项目邀请函}

\begin{frame}
	\begin{dfn}[§ 1.2]
		项目邀请函(2024年5月)
	\end{dfn}

\end{frame}

\begin{frame}
	\textit{走过春的旖旎,迎来夏的缤纷,时光清浅,不知不觉间大学一年级的学习时光接近尾声,2024高考也将如约而至。}
	\vspace{0.5cm}
	
	\textit{无意间点开互联网上一些关于「 高考 」的讨论,看见无数个“XX大学等我”,恍惚间,逝去的时光又回来了,搁置的热情好像又有了清晰的模样。}
	\vspace{0.5cm}

	\textit{曾经平凡的一切,在有一天也成为了遥远的故事......}
	\vspace{0.5cm}

	\textit{或许迷茫与徘徊,或许还在等待梦想。}
\end{frame}


\begin{frame}
	\textit{大学四年已将要度过一年,可能在不是那么遥远的未来,我们也会穿上学士服在学校留影纪念,我们也会像怀念中学时光那样怀念我们的大学时光。}
	\vspace{0.5cm}

	\textit{我们都会希望在三年后的那个岁月关口,能更加自豪地回首当下我们正在经历的一切。
	或许已有方向,或许还在路上。}
	\vspace{0.5cm}

	\textit{不久以后,会有一群同学像曾经的我们一样叩入武科大的校门。
	面对未知,他们一定也很迷茫。}
	\qed
\end{frame}


\subsection{手册说明}
\begin{frame}
	\begin{dfn}[§ 1.3]
		手册说明
	\end{dfn}
\end{frame}

\begin{frame}
	Survive-WUST-Manual(武汉科技大学生存手册)项目灵感来自于Survive-SJTU-Manual(上海交通大学生存手册),旨在帮助初入校园的同学更快地适应大学生活,完成从高中生到大学生的身份转变。
	\vspace{0.5cm}

	在 Survive-WUST-Manual 中,我们将多维度探讨大学生活的可能性,辅助读者的价值判断与生涯规划。
	\vspace{0.5cm}

	<武汉科技大学生存手册>由一群武科大本科生(计算机学院学生为主)写就,手册写作灵感来自于上海交通大学生存手册。本手册与上交大版本不同之处在于——\textbf{本手册是依据武汉科技大学真实情况写就的,对原版本中的许多内容进行了适用性修改。}
	\vspace{0.5cm}

	由于篇幅有限,更详细的说明及更新日志请参考手册的官方网站
	\vspace{0.1cm}
	www.survivewustmanual.cn 。
	\qed
\end{frame}

% 立志篇
\section{立志篇}
\subsection{欢迎来到武汉科技大学}



\begin{frame}
	\begin{dfn}[§ 2.1]
		欢迎来到武汉科技大学
	\end{dfn}

	\begin{ex}[]
		@ 何朝晖 @ 祁仁政
	\end{ex}
\end{frame}

\begin{frame}
	恭喜你们乘着六月的风,历经高考的试炼成为武科大的一员!
	\vspace{0.5cm}
	
	回首望,轻舟已过万重山。向前看,前路漫漫亦灿灿。
	\vspace{0.5cm}

	在这里,你们将开启一段崭新的旅程。走进静谧典雅的图书馆,漫步微风拂面、绿柳成荫的沁湖,更有诗会和鱼宴等待着你们到来。
	\vspace{0.5cm}

	三月的樱花雨,十二月的沁湖鱼,图书馆的时钟,教三楼的灯光......都会成为你大学生涯的宝贵回忆。

\end{frame}

\begin{frame}
	高考有标准答案,但是人生无法被定义,开拓探索是大学生活的主旋律——
	\vspace{0.5cm}
	
	\textbf{从迈出高考考场的那一刻起,人生就从一道道选择题变成了一道道开放题,我们的生命不再有唯一的答案,成功也不只有一个固定标准。}
	\vspace{0.5cm}
	
	\textbf{高考以分数论高低,但是每一个人的人生价值无法靠成绩来定义。独一无二的自己,才是值得用一生去追寻的宝藏。}
	\vspace{0.5cm}
	
	所以,请带着勇敢和热爱出发,开启属于你的大学生活!衷心祝愿你们能够在这里度过充实而有意义的四年时光。
	\qed
\end{frame}

\subsection{跳出 考败来科 的思维怪圈}

\begin{frame}
	\begin{dfn}[§ 2.2]
		跳出 ⌈考败来科⌋ 的思维怪圈
	\end{dfn}

	\begin{ex}[]
		@ 何朝晖 @ 付子昂
	\end{ex}

\end{frame}

\begin{frame}
	\begin{LARGE}
		\textbf{遗憾、迷茫 and 失落}
	\end{LARGE}
	
	\vspace{1cm}

	\begin{Large}
		热衷考取功名,这是中国人的传统。
	\end{Large}
	\vspace{0.75cm}

	时光荏苒,辗转过细,十多年来一直辛苦付出的少年,怀揣着成人成才的梦想去一次次努力地拼搏,一定也是希望用自己的努力换取一份不错的成绩。
	\vspace{0.5cm}

	人们对于考取学校的期望往往是比较高的,以至于很有可能满足不了自己的期望,让高考成为遗憾。

\end{frame}



\begin{frame}
	\begin{Large}
		遗憾的造就
	\end{Large}
	\vspace{0.75cm}

	武汉科技大学的分数线是比较高的,超过一些211的某些专业。
	\vspace{0.5cm}

	以至于每一年总会有学生在来校后后悔自己没有选择其他院校,而是选择了武汉科技大学。
	\vspace{0.5cm}

	看到相同分数段的同学进入“更好”的学校,往往会感到些许失落。
	\vspace{0.5cm}

	这些都是正常的心理,但是带着这种心理往往会限制自我发展。
\end{frame}

\begin{frame}
	\begin{LARGE}
		\textbf{正确认识高考成绩}
	\end{LARGE}
	\vspace{1cm}

	决定上限的是水平而非文凭。高考成绩固然重要,但比成绩更重要的,是备战高考的过程,教会了我们学习的能力,带给了我们精神的成长,开阔的视野在求知中建立,不屈不挠的性格在锤炼中养成。倘若高考没有达到心中的预期,绝不意味着自己就是不行的。保持刻苦耐劳的品格,养成终身学习的习惯,勤于在实践中磨练,提升发现问题、解决问题的本领,永远热血沸腾永远满怀期望,不抛弃不言弃,这些都是比一纸文凭更为重要的,是安身立命、行稳致远的底气所在。
	\vspace{0.5cm}

	\textbf{高考考分既不是实现自我价值的充分条件,也不是必要条件。}学生时的分数,成年后的收入,人生只在追求功利的数字,大概也很没有意思。
	

\end{frame}


\begin{frame}
	\begin{LARGE}
		\textbf{贪心算法}
	\end{LARGE}
	\vspace{1cm}

	\textit{贪心算法(Greedy Algorithm)是指,在对问题求解时,总是做出在当前看来是最好的选择。}
	\vspace{0.5cm}

	\begin{itemize}
		\item 不可否认的是,\textbf{贪心} 是刻在人DNA里的一种 \text{性格}。
		\vspace{0.25cm}

		\item \textbf{贪心算法} 是一种在每一步选择中都争取当前状态下最优的解决方案,以期望达到全局最优解的算法。这种算法的特点在于其逐步构建最终解决方案的方式,在每一步都做出当前可用的最佳选择,而不考虑这些选择可能对问题全局解的影响,因此,贪心算法在许多情况下能够提供局部最优解,但不一定能保证得到整体最优解。
		
	\end{itemize}

\end{frame}

\begin{frame}
	\begin{itemize}
		\item 贪心算法的关键在于贪心策略的选择,不同的策略可能导致不同的局部最优解,进而影响最终的结果,这种算法适用于那些具有最优子结构和无后效性的问题。在背包问题、最短路径问题等许多领域,贪心算法都能发挥重要作用。
		\vspace{0.25cm}

		\item 其中尤其重要的一点是——\textbf{贪心算法的局部最优解未必是全局最优解。}
		\vspace{0.25cm}

		\item 也就是说,\textbf{可能当下这个你填报志愿时看起来并不好的选择,可能是你人生里最优的选择。}
	\end{itemize}

\end{frame}

\begin{frame}
	假定你凭借你的分数上了985或211的冷门专业,很有可能大学本科四年都沉浸在学习不感兴趣的知识的痛苦中。并且显而易见的是,那些院校的学业压力是要比武科大大上一个梯度的。而在这里,你可以在学业上更加游刃有余,有更多的空余时间用于提升自己。
	\vspace{0.5cm}

	\textbf{在你羡慕别人的时候,或许别人也在羡慕你。}
\end{frame}

\begin{frame}
	\begin{LARGE}
		\textbf{统计学规律与自我实现的预言}
	\end{LARGE}
	\vspace{1cm}

	\begin{itemize}
		\item 自我实现的预言也称“自证预言”,实际上是自己给自己潜意识贴标签。
		\vspace{0.25cm}

		\item 人会不自觉地按预言行事,最终令预言发生。这个的预言其实是你对事情的看法。
		\vspace{0.25cm}
		
		\item 而预言的乐观与悲观,都取决于我们自己。
	\end{itemize}

\end{frame}

\begin{frame}
	\textbf{“二本坐过牢,双非有案底。”}这只是互联网上的调侃,这也一方面是统计学总结出的客观规律——双非本科就业处于相对不利地位。
	\vspace{0.5cm}

	Of course,随口自嘲一下也没有什么问题。但是如果心底真正相信这句话,往往会在无形中\textbf{给自己设限},以至于最终\textbf{预言成真},真的处处碰壁,颠沛流离。
	\vspace{0.5cm}

	统计学规律可能会告诉你,你的选择有多么多么不好。但是请记住,\textbf{统计学用数据提炼的规律足够科学,却无法框住你,你不止是一个随机样本,能带你冲破概率的,是你自己。}
	\qed
\end{frame}

\subsection{摈弃陈旧的线性思维}
\begin{frame}
	\begin{dfn}[§ 2.3]
		摈弃陈旧的线性思维
		\vspace{0.25cm}

		---\space 修改中 \space 待上线 ---
	\end{dfn}

	\begin{ex}[]
		@ 付子昂 @ 何朝晖 @ 鲁昊哲
	\end{ex}
\end{frame}



\subsection{探索真正想做的事}

\begin{frame}
	\begin{dfn}[§ 2.4]
		探索真正想做的事
	\end{dfn}

	\begin{ex}[]
		@ 何子瑞
	\end{ex}
\end{frame}

\begin{frame}
	\begin{LARGE}
		\textbf{All of this is changing, but we seem to be ignoring it.}
	\end{LARGE}
	\vspace{1cm}

	想必各位同学在高中三年的学习中,已经在作文里习惯于以\textbf{“21世纪是一个充满机遇与挑战的世纪”}作为开头。但实际对于大部分“小镇做题家”而言,是感受不到所谓“大背景”的影响。但你应该有过印象,从小学起街边5毛钱一张的薄脆馅饼,一元一瓶的冰露矿泉水。到如今五元一张的原味馅皮,各种各样起步价就是几十元的“精美奶茶”。
	\vspace{0.5cm}

	这些商品的变迁与价钱的改变,似乎在告诉着我们,是不是在我们看不见的地方,有什么东西,正随着时间在暗流涌动,甚至于渐渐浮出水面,极大地影响我们的生活。
\end{frame}

\begin{frame}
	\begin{LARGE}
		\textbf{我们所处的大背景}
	\end{LARGE}
	\vspace{1cm}

	如果真的要我去形容,对于绝大部分人而言,我们正处在一个\textbf{“后真相时代”},正处在一个\textbf{“被互联网和传统语境定义话语权与价值”}的年代。
	\vspace{0.5cm}

	什么是“后真相时代”,相信同学们在高考前一定做过历年的高考真题,2023语文新高考一卷的现代文阅读的第一篇文章,便摘自赫克托·麦克唐纳的《后真相时代》。抛开这篇文章作为高考题目的身份,它也是一段能够极好阐述所谓“后真相”含义的文字。“后真相”简单而言:\textbf{在当下国内外各种矛盾突出的世代,人们在乎的其实并不是严肃的事实,而是在寻求一种情绪上的爆发、情感的共鸣。}
\end{frame}

\begin{frame}
	正如《经济学人》在其发布的“Art of the lie”一文中对后真相进行解读——\textbf{“真相没有被篡改,也没有被质疑,只是变得次要了。”}
	\vspace{0.5cm}

	但需要明确的是——\textbf{后真相并非指真相缺席,而是指真相被认知的过程更为崎岖,人们能接受谎言(虚假信息)而对待真相的态度也更为平淡。}
	\vspace{0.5cm}
	
	受众不在意信息是否为真,他们更为在意的是信息背后的主张所包含的情感因素,当受众将自身想表达、宣泄的情绪发出后,便不再顾及舆论事件的后续发展便纷纷离开舆论广场。而\textbf{互联网}则为这种“后真相”提供了极好的土壤。人们在上面发声,搜集信息的成本越来越低。取而代之的则是毫无思考的“凑热闹”。于是在这种\textbf{刻意}的宣传和发酵下,我们能搜集到的,能耳闻到的,都是“情绪的堆叠”,都是一场虚假的宣传,乃至于狂欢。
\end{frame}

\begin{frame}
	而作为一个初出茅庐的大学生而言,我们几乎没有任何机会去核查事实,一但我们失去辨识能力。我们便会逐渐按照别人为我们所定义的“成功”,丢失于属于自己的\textbf{话语权},走向如同傀儡般的人生,再也没有自己的一席之地,而\textbf{自我意识}也被遗失在了某个荒凉之处,永远不见天日。
\end{frame}

\begin{frame}
	\begin{LARGE}
		\textbf{培养自我意识}
	\end{LARGE}
	\vspace{1cm}

	上面的一段话对于某些同学来说可能是危言耸听,甚至于感到不以为意。但是请你好好思考,为什么我们在立志篇·第二部分会强调如何跳出“考败来科”的思维,在那一部分我们只教方法,但是我们没有讨论为何会存在这种思想。似乎在我们的心底从一开始就预设了一种“高考没考上211 985就是相对失败”的想法。
	\vspace{0.5cm}

	回到上面那句“我们正处在一个\textbf{被互联网和传统语境定义话语权与价值的年代}”,我们似乎只谈论了互联网与后现代的关系,但我们忽略了一个词“\textbf{传统语境}”。
\end{frame}

\begin{frame}
	你好好回想一下,是什么时候我们拥有了这种“\textbf{学历优势论}”呢?
	\vspace{0.5cm}

	是从漫天飞舞的信息,身边人的见闻,老师朋友的口中吗?好像是?但又不是。明明学习只是为了让我们过上更好的生活,为了成功。可什么时候变成了“高考分数低等于失败,考不上好大学就应该被人厌恶”呢?什么时候变成了“\textbf{学历绝对论}”呢?再更进一步往下想想,为什么一定要\textbf{成功}呢?\textbf{成功}到底是意味着什么呢?别人口中所强调的\textbf{成功},是你想要的\textbf{成功}吗?
	\vspace{0.5cm}
	
	如果你看了这段话若有所思,甚至于曾经思考过这些问题。恭喜你,你已经半步踏入自我意识的领地了。但是我们仍然没有解决问题,我们只提出了问题,甚至于问题的提出,你可能也是从别的地方看到。
\end{frame}

\begin{frame}
	自我意识?这是个模糊的概念。而且关于“\textbf{什么是自我意识}”这个问题的讨论也过于复杂,古今中外也有许多大家著书立说讨论过这个问题。所以在这里我们先避开自我意识本身的讨论,我们先来讨论\textbf{如何培养自我意识}。
	\vspace{0.5cm}

	落实到个人上,我认为还是\textbf{思考}和\textbf{怀疑}。我们要怀疑周围人是不是对的,也要怀疑自己是不是对的。这是一个痛苦的过程,这意味着你要否定自己相信的理论(虽然对于大部分人甚至于没有形成自己完整的三观),甚至否定你自己。\textbf{你将不会再有一个安心依靠,无条件相信,并从中获得信心与认可的东西}。你将会直面他人的恶意,因为拥有自我反省和独立思考的人终究是少数。你可能会被误解,被孤立,甚至遭受各种渠道上的攻击。\textbf{但是只有当你的心中燃起理性的火焰时,你才真正算是向人类完成了进化。}
\end{frame}

\begin{frame}
	不然你便和占据绝大部分的随波逐流者一样,只是报团取暖的猴子罢了。你如果停止了思考,停止了怀疑,选择拥抱空气(这里可以搜索\textbf{读空气}。是一种在日本的人际交往用词),那么你的思考将不再是你的思考,你的梦想将不再是你的梦想(比如考大学的梦想,真的属于你吗?),你的观点将不再是你的观点,你的人生价值将不再是你的人生价值。
	\vspace{0.5cm}

	这些东西可以是空气,周围人,互联网,乃至于任何人赋予给你的,但唯独不属于你自己。你将失去作为独立个体的存在,而作为一团巨大空气的一个分子。
\end{frame}


\begin{frame}
	但是,幸运的是,我认识到这一切的时候,我告诉自己——现在开始思考,永远都不会太迟。同学们也一样。
	\vspace{0.5cm}

	怀疑与思考永远是人类的根基,是自我的必需。
\end{frame}

\begin{frame}
	\begin{LARGE}
		\textbf{做自己想做的事}
	\end{LARGE}
	\vspace{1cm}

	在当今的世界之中,互联网看样子是让人群走向了分化,毕竟我们能够接触到更多的信息了,自然也就会有比以前种类更多的颜色了。
	\vspace{0.5cm}
	
	然而,事实上,互联网极大的降低了交流成本的结果,反而是让大家更进一步丧失了独立思考的能力,让大家更依赖于空气,因为在交流成本降低后,利用舆论与水军,左右人们观点,影响整体空气的成本便也同时降低了。许多人更容易被牵着鼻子朝着某些人所期望的方向走去,而也是因为互联网极大的降低了交流的成本,它也同时极大的降低了暴力的成本。
\end{frame}

\begin{frame}
	资本,媒体,或者是其他居心叵测的人,会通过空气来获利,他们又会反过来以金钱的力量,一遍又一遍的传播和加固着这种空气,最后甚至演变成政治正确。通过符号暴力(俗称“戴帽子”)来侮辱,打压那些正确的人(比如“\textbf{你这个双非的学生}”)。而渠道方,也会通过各种平台上的人工智能算法,反复地向你传播和加固这些内容,通过这种方式让你深信不疑,美其名曰“垂直度”。
	\vspace{0.5cm}

	所以也就像我上面说的,拥有\textbf{独立思考和怀疑}的能力的人,自然也就是少之又少了。但好在人类的历史上从来不缺乏敢于打破空气,打破沉默,直面暴力的人。我们有苏格拉底,有布鲁诺,有鲁迅,但我想说,更重要的是,我们要培养自己的独立意识,我们要做自己想做的事,我们要成为\textbf{我们自己}。
\end{frame}

\begin{frame}
	说了这么多,我认为培养思考和怀疑的最直接方式还是阅读。
	\vspace{0.5cm}

	看100个短视频不如读一本好书,虽然读书也算是个一直在被强调的事。但细细想来,有多少人在自己的18个春秋中完整的读完过一本著作并从中有所感悟的呢?我想很少。而在大学里,你没有过于繁重的学业。你完全可以通过大量的阅读感受世界,寻找自己。在这里我推荐马克思和恩格斯的《资本论》和周国平的《守望的距离》,我不想在这里赘述这两本书适合什么样的群体,读完这本书你能获得什么,因为这些东西本该由你自己去定义。我相信,沉下心来,认真思考,会有收获,会有改观。
\end{frame}

\begin{frame}
	关于副标题“\textbf{做自己想做的事}”,我认为在大学的环境下,这句话尤为重要。但是重要归重要,任何一句话如果没有合适的语境和前提都是白扯。为什么一开始强调自我意识,因为一个人最重要的就是精神建设,没有精神建设也就没有思想,没有属于自己的思想便不存在\textbf{真正的“自己想做的事”}。基于自己的思考,才能得出最终的答案。给出我自己对于这个副标题的一些看法吧,不然没有太多的说服力。我会给出我的些许思维方式,帮助你们去更好的认知世界.
\end{frame}

\begin{frame}
	\begin{enumerate}[1]
		\item 认识词语,无论任何词语,只要存疑。可以通过互联网搜索释义。我更希望同学们能重新去搜索曾经无数次见过的词语,熟悉的才是陌生的。
	\end{enumerate}
	\vspace{1cm}

	\textit{eg:“大学”,一个被我们无数次提起但被无数人忽视的词语。接下来我将以这个词展开讨论。在新华词典中,“大学”古指聚集在特定地点传播和吸收高深领域知识的一群人的团体;现在指提供教学和研究条件和授权颁发学位的高等教育机关。}
	\vspace{0.5cm}

	相信很多人是第一次看到这个词的官方含义吧,这意味着你对世界迈出了主动认知的第一步。
\end{frame}


\begin{frame}
	\begin{enumerate}[2]
		\item 根据自己的个人经历,包括接触到的信息(更多参考书籍,名家)。重新定义这个词语
	\end{enumerate}
	\vspace{1cm}

	\textit{eg:这里我想借用一段蔡元培先生对于大学的理解。大学者,”囊括大典,网罗众家”之学府也。1917年蔡元培在北大就职演说上又提出“\textbf{大学者,研究高深学问者也}”。他认为“\textbf{治学者可谓之大学,治术者可谓之高等专门学校}”。这里由于篇幅原因,故不赘述。感兴趣可移步《就任北京大学校长之演说》。}
	\vspace{0.5cm}

	随着自己认知的加深和经历的增多,许多词语的定义会在我们心目中不断变化,变成我们世界观尤为重要的一部分。
\end{frame}


\begin{frame}
	\begin{enumerate}[3]
		\item 提问
	\end{enumerate}
	\vspace{1cm}

	\textit{eg:人在定义词语的过程中,难免会遇到自己不懂的,可能是出于词义的陌生,但更多是对于未知的不解。我们会产生许多的问题。我们为什么要上大学?上大学读书的目的是什么?大学存在的目的是什么?这些问题我希望同学们自己去探索。}

\end{frame}

\begin{frame}
	\begin{enumerate}[4]
		\item 经历,实践,认知,改变。然后重复2和3的过程。
	\end{enumerate}
	\vspace{1cm}

	\textit{eg:上大学的第一天,你会发现这种教学模式也好,人与人之间的关系也好,都可能与你先前的定义不符。那么不用怀疑,大胆改观,改变你的定义与认知!接受你眼前所见的,但有时也要记住,适当的思考会避免你过于相信眼前看到的事物。尝试开阔自己的眼界。一个终日在寝室里打游戏的大学生是不会觉得大学生活是有趣的,但这不意味着大学生活无聊。接受但又不完全接受,时刻保持思考,相信眼前的一切,但是又时刻保持怀疑,这很矛盾,但这才是我们保持自我的唯一方式。}
\end{frame}


\begin{frame}
	最后的最后,倒是感觉自己挺啰嗦的。中间也会有许多“\textbf{只可意会不可言传}”的部分。但我希望同学用自己的方式,去定义,去发现,去思考“\textbf{自己真正想做的事}”这段话,养成属于自己的三观。如果你在进入大学前,就已经明确了自己想做的事,请你坚持。但如果你没有,也不要紧。当你真正通过自己的思考,结合自己的经历。找到了属于自己的山岳,请你勇敢的去攀登吧!大学很短,我们时间不多;人生很长,我们来日方长。走自己的路,让别人说去吧!

	\qed	
\end{frame}
% Applications

\section{访谈篇}
% Application 1




% Conclusion
\section{生存技巧}


\end{document}